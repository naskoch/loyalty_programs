\documentclass{article}

\usepackage{graphicx}
\usepackage{algorithm2e}
\usepackage{float}
\usepackage{amsmath}

\begin{document}

We have reduced the problem of revenue maximization (for firm $A$) under proportional budgeting to the following maximization problem.
\begin{equation}
\max_{\alpha} \left\{(1-\alpha v)E_{\lambda} \left[p \frac{\lambda}{1-\frac{\alpha}{e}(1-\lambda)}+(1-p)\lambda\right] \right\}
\end{equation}

Furthermore, because any reward scheme may be written as a proportial budgeting scheme for appropriate $k$ and $\alpha$ values, the above optimization also gives results beyond just proportional budgeting. The optimization problem may be written as:
\begin{equation}
\max_{\alpha} \left\{(1-\alpha v) \int_0^1 \left(p \frac{\lambda}{1-\frac{\alpha}{e}(1-\lambda)}+(1-p)\lambda \right) f(\lambda) d\lambda  \right\}
\end{equation}
where $f(\lambda)$ is the pdf of $\lambda$, supported on $[0,1]$.

One method to evalute the above was to use the Taylor series expansion of moments of random variables. Let $g(\lambda)$ be the function in the integrand above (exluding the pdf), so in our case, the method reduces to:
\begin{align*}
\max_{\alpha} \left\{(1-\alpha v) E_{\lambda} (g(\lambda) \right\} &\approx \max_{\alpha} \left\{(1-\alpha v) \left(g(\mu_{\lambda})+\frac{g''(\mu_\lambda)}{2} \sigma_{\lambda}^2 \right) \right\}
\end{align*}
However, I'm not sure what conditions on the distribution we need to ignore the higher order terms of the series expansion.

First I want to examine the problem with simple delta distributions, i.e. $\lambda$ constant. There are three cases we care about.

\begin{enumerate}
\item ($\lambda = 0$) Here we are maximizing $(1-\alpha v)\left(p \frac{0}{1-\frac{\alpha}{e}} \right)$. This function is 0 for all values of $\alpha$ (even as $\alpha \rightarrow 0$). Thus, this result shows (matching our intuition) that some excess loyalty is needed for anyone to adopt the reward program in the model - just another way to think about it.

\item ($\lambda = 1$) Here we are maximizing $(1-\alpha v) \lambda$, which occurs at $\alpha = 0$. Here again we just get an obvious result: if all customers have excess loyalty 1, the firm will get all business and should not offer any reward scheme.

\item ($0 < \lambda < 1$) I have been messing with this for a bit. It is quite a messy expression. Through plotting, I've seen some that are maximized at $\alpha \rightarrow 0$ and some at $\alpha \rightarrow e$ (generally when $\lambda$ is large and small, respectively). I have yet to see conditions that give $\alpha$ something in between. 
\end{enumerate}

I figure this may be the easiest problem we can think about solving, so this is what I've been trying to do first.

\end{document}