We investigated the optimal design of a frequency reward program against traditional pricing in a competitive duopoly.
We modeled the behavior of customers valuing their utility in rational economic terms, and our theoretical results agree with past empirical studies.
Assuming general distributions of customer population, we characterized optimal parameters for the design of reward program, and under more specific parameter distrubution assumptions, we showed the conditions on customer population parameters which make the reward program strictly better.
In short, if a merchant can make good estimates of the customer population parameters, our model and results can help understand the pros and cons of running a frequency reward program for that merchant against traditional pricing.

Though our research offers some interesting managerial insights, there are some limitations to our study. 
Our results on revenue comparisons assumed specific distributions for the customer population, though our framework can be extended to other distributions as well. Moreover, estimating the customer population distribution and parameters using real transactional data is an interesting question in itself.
That is, backing this research with empirical and experimental study, could provide strong quantifications to the intuitions we discuss.
We modeled customer behavior in rational economic terms, mainly to understand the rational components that affect the decision making process.
Tying in the effects of our research with some past models on psychological behavior patterns of customers toward reward programs would be another practically relevant problem to address.
Finally, we modeled a competitive duopoly, but left the traditional pricing merchant as non-strategic.
Understanding how competition affects the equilibrium prices and reward program parameters could give intuitions about a more practical scenario. 
