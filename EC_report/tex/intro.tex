Loyalty programs constitute a huge market in consumer retail and are a major source of revenue for many low margin businesses.
Over 48 billion dollars in perceived value of rewards is issued in the United States alone every year, with every household having over 19 loyalty memberships on average (\cite{berry2013loyalty}).
This market constitutes credit cards, hotel and airline reward programs, and more recently even restaurants, grocery and retail stores.
In addition to possibly increasing their market share, these reward programs provide many benefits to the merchants -- for instance, user identification for firms having multiple purchase channels; increase in sales due to referrals; personalized price discrimination and product recommendations, to name a few (\cite{ryu2007penny}).
There are many examples of popular reward programs -- Starbucks allows members to earn ``stars'' on purchases which can be redeemed for free coffee, Bloomingdale's offers \$25 reward for around \$1500 spent in their store, and Target offers a 5\% cashback on all purchases (\cite{cvs2015target}).
Though forming a big component of the market, there is little scientific understanding about the design of loyalty reward programs. We aim to address this gap with our research.

One popular form of loyalty reward programs is \emph{frequency reward programs}, where customers earn \emph{points} as currency over spendings with merchants and are able to redeem these points for dollar valued rewards after achieving certain threshold point collections.
There is extant literature on characterizing customer behavior toward frequency reward programs.
Most of the literature is empirical in nature, and relies on psychological behavioral patterns among customers, as opposed to rational economic decision making (\cite{kivetz2006goal,dreze2004using,gao2014influence}).
In this paper, we consider a competitive duopoly of two merchants where one merchant offers a frequency reward program and the other offers traditional pricing with discounts.
Though revenue management literature often deals with dynamic pricing of products, many retail merchants offering rewards often pre-commit to their prices. 
We assume that both merchants commit to their product pricing apriori and characterize a novel model of customer choice where customers measure their utilities in rational economic terms.
In addition, we investigate the direct effects on the revenue objective and characterize the optimal reward design choice for the merchant offering the frequency reward program, based on different customer populations.
Specifically, we answer the following question: how should the merchant decide the optimal thresholds and dollar value of rewards to optimize for its revenue share from the participating customer population.
One important constraint we impose is that the merchant has to choose a \emph{one design fits all} reward program for the entire participating customer population and is not allowed to personalize the program for different customer segments.

This is how the remaining paper is structured. First, we will describe some related work. Then we will give an overview of our model and results. 
In Section~\ref{sec:model}, we will describe our model in technical detail followed by the main results in Section~\ref{sec:results}. 
We will conclude with a short discussion on future work in Section~\ref{sec:conc}.

\subsection{Related Work}
Three popular psychological constructs have been used to explain customer choice dynamics toward reward programs -- Goal Gradient Hypothesis, Medium Maximization, and Tipping Point Dynamics.
\cite{kivetz2006goal} conducted an empirical study observing an acceleration in the number of purchases by customers as they approached the reward, \ie, as customers accumulated reward points to reach closer to achieving the reward, their effort invested toward gaining more points increased. 
The authors attributed this behavior to Goal Gradient Hypothesis (\cite{hull1932goal}). 
This behavior is also very prevalent in online badge systems, such as those on Stackoverflow; recently, mathematical models relying on rational user behavior have been developed that explain this phenomenon (\cite{leskovec2013steering}).
\cite{stourm2015stockpiling, dreze2004using} observed that customers often stockpiled reward points even when there were economic incentives against the collection of points. They attributed this behavior to Medium Maximization -- customers often treated collecting reward points as a goal itself just like collecting stamps as opposed to connecting reward points with economic incentives.
Correspondingly, they introduced a new model where customers had different ``mental accounts'' and utility functions for points and cash.
\cite{gao2014influence} observed via experimentation that customers often collect reward points for exogenous reasons until they accumulate a threshold amount, after which they start investing effort toward the collection process itself.
That is, customers build up switching costs (\cite{klemperer1995competition}) before fully adopting the reward program, and sometimes this switching cost is created due to reasons exogenous to rational economic incentives.
They referred to this behavior as the Tipping Point Effect.

A large body of literature investigates the switching costs customers face within a competitive duopoly framework -- see \cite{villas2015short} for a short survey.
Our model is closest in spirit to that of \cite{hartmann2008frequency} and \cite{kopalle2001economic}.
Both papers are empirical in nature and model a competitive duopoly where customers maximize their long term discounted utility.
\cite{hartmann2008frequency} argue that less frequent buyers face higher switching costs as they are more likely to be affected by reward redemption deadlines, whereas frequent buyers redeem rewards easily and do not face substantial switching costs. 
They do not model how customers build up switching costs, but only argue what happens when customers are close to achieving a reward.
\cite{kopalle2001economic} discuss dynamic competition between two merchants deciding whether to offer a reward program or traditional pricing and model this decision problem as a two stage game: first merchants decide whether to offer a reward program or traditional pricing and then they decide their prices. 
Using simulations, depending on customer parameters in the model, they characterize the conditions for when it is better to offer a reward program versus traditional pricing.
We on the other hand model a multi-period problem where the customer behavior is characterized using a complete dynamic program, and mathematically analyze our model.
We make two modeling assumptions: first is an exogenous visit probability bias toward the reward program merchant which can be attributed to \emph{excess loyalty} -- customers often build up higher brand preference toward the merchant offering a reward program (\cite{fader1993excess, sharp1997loyalty}); 
and second, a look-ahead factor for customers, which indicates how far into the future customers can perceive the rewards (\cite{liu2007long,lewis2004influence}).
Our results on customer choice dynamics intuitively look similar to some of those obtained in the above mentioned body of literature.
But more importantly, we model and optimize the revenue objective of the merchant, characterizing an optimal reward program design for maximizing expected revenue.


\subsection{Our Contributions}
We make both modeling and analytical contributions along the following directions:

\subsubsection{Competitive Duopoly Setup}
We model a competitive duopoly of two merchants, one of them offering a frequency reward program and the other offering traditional pricing.
Both merchants sell an identical good at fixed precommitted prices.
The reward program merchant sells the good at a higher price.
Customers are drawn from a known population distribution and they measure their utilities in rational economic terms, \ie, they make their purchase decisions to maximize long term discounted rewards.
The discount factor is the time value of money, and is constant for all customers.
Every customer makes a purchase everyday from either of the two merchants.
With each purchase from the reward program merchant, customer gains some fixed number of points, and on achieving the reward redemption threshold, (s)he immediately gains the reward value as a dollar cashback.

\subsubsection{Customer Parameters}
Each customer has two important parameters drawn from the known population distribution: first a visit probability bias with which (s)he purchases the good from the reward program merchant for reasons exogenous to utility maximization, and second a look-ahead factor that controls how far into the future the customer can perceive the rewards.
The visit probability bias toward the reward program merchant can be attributed to \emph{excess loyalty} which has been argued as a important parameter for the success of any reward program, or it can be attributed to price insensitivity of the customer; whenever the customer is price insensitive, (s)he strictly prefers to purchase from the reward program merchant as (s)he gains points redeemable for rewards in the future.
There are many possible reasons for customers' price insensitivity: the reward program merchant could be offering some other monopoly products, or the customer might be getting reimbursed for some purchases as part of corporate perks (eg: corporate travel).
As an effect, this visit probability bias controls how frequently the customers' points increase even when (s)he does not actively choose to make purchases from the reward program merchant.
The look-ahead parameter affects the customer behavior dynamics as follows: if the reward is farther than the customer's look-ahead parameter, (s)he is unable to register the future value of that reward and take it into consideration while maximizing long term utility.
Both these parameters can be attributed to bounded rationality of customers and have been argued to be important factors toward customer choice dynamics.


\subsubsection{Customer Choice Dynamics}
We formulate the customer choice dynamics as a dynamic program with the state being the number of points collected from the reward program merchant.
When the customer does not make biased visits to the reward program merchant, she chooses the merchant maximizing her long term utility: comparing the immediate utility by purchasing the good at cheaper price and the long term utility of waiting and receiving the time discounted reward.
The solution to the customer's dynamic program gives conditions for the existence and achievability of a phase transition: a points threshold before which the customer visits the merchant offering rewards only due to the visit probability bias, and after which (s)he always visits the merchant offering rewards till receiving the reward.
We show that this phase transition point has the following dependencies:
\begin{enumerate}
\item It decreases with increase in the look-ahead parameter, \ie, if the customer can perceive rewards longer into the future, the phase transition for the customer occurs sooner.
\item It decreases with increase in the reward value, and decreases with increase in points threshold required to redeem the reward. 
\item It increases with the discount value that the traditional pricing merchant provides, \ie, the cheaper the product from the second merchant, the farther is the phase transition point. 
\item With respect to the discount factor there is a unique minimizer for the phase transition point. That is, if customers are highly patient, then maximizing their long term utility leads them to redeem rewards only due to exogenous visit bias.
And if they are very less patient, they strictly prefer immediate discount over future rewards.
\end{enumerate}

We define the ``influence zone'' as the ratio of the distance to the phase transition  point and the total distance to the reward.
This is effectively the fraction of total purchases until reward redemption that the customer needs to make due to exogenous visit bias to adopt the reward program for future purchases.
Merchants often give bonus miles or points to their customers, mainly to increase the switching cost of customers and push them to adopt the reward program.
The influence zone as we defined is the point till such external promotions are required.
We optimize over this influence zone and show that the optimal distance to the reward has a very intuitive form which corresponds closely to practically observed values for the reward distance.

In short, these results verify that our model is in coherence with the different psychological constructs as discussed in the related work section: purchase acceleration closer to reward redemption and a tipping point before which purchases are only due to exogenous visit bias.

\subsubsection{Reward Program Design}
After characterizing the customer behavior dynamics in our model, we optimize over the long run revenues that the reward program merchant achieves.
We provide a framework for optimizing the parameters for the reward program merchant to maximize its revenue objective.
We then look at a specific simplified case of proportional promotion budgeting: the reward offered by the reward program merchant is proportional to the product of the distance to the reward and the discount provided by the traditional pricing merchant.
We show that under proportional promotion budgeting the optimal distance to reward for maximizing the revenue objective of the reward program merchant is independent of the customer parameters and is close to the real world obtained cashback percentage amounts.
In addition, optimizing the revenue objective gives the same optimal distance to reward as optimizing the influence zone as defined above.
Moreover, we characterize the conditions in terms of the customer parameters for when the revenue objective of the reward program merchant is better than the traditional pricing merchant and when it is better for the reward program merchant to offer a reward versus not offering any reward.
We show that for the reward program to be effective under both the above conditions, a minimum threshold fraction of customers must be ``forward-looking''.
And there is a range of values of the exogenous visit probability bias between $0$ and $1$ corresponding to the fraction of ``forward-looking'' customers for the reward program to be strictly better for the merchant. 

