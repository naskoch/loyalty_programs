\subsubsection{Model Overview}
We model a competitive duopoly of two merchants, one of them offering a frequency reward program and the other offering traditional pricing.
Both merchants sell an identical good at fixed precommitted prices.
The reward program merchant sells the good at a higher price.
With each purchase from the reward program merchant, a customer gains some fixed number of points, and on achieving the reward redemption threshold, (s)he immediately gains the reward value as a dollar cashback.

Customers measure their utilities in rational economic terms, \ie, they make their purchase decisions to maximize long term discounted rewards.
The discount factor is the time value of money, and we assume it to be constant for all customers.
We also assume that every customer makes a purchase everyday from either of the two merchants.
We relax these two assumptions by introducing a look-ahead factor that controls how far into the future a customer can perceive the rewards. 
This affects the customer behavior dynamics as follows: if the reward is farther than the customer's look-ahead parameter, (s)he is unable to perceive the future value of that reward and take it into consideration while maximizing long term utility.
This parameter, being customer specific, adds heterogeneity to both the future discounting and purchase frequency.
We only model myopic and strategic customers, \ie, the look-ahead parameter being $0$ or a large value, and leave further parametrization for future work.
{\nolan I'm looking for source about different levels of myopicity and then I will make this a little stronger.}
In addition, we assume each customer has a visit probability bias with which (s)he purchases the good from the reward program merchant for reasons exogenous to utility maximization.
This can be attributed to \emph{excess loyalty} (\cite{fader1993excess, sharp1997loyalty}) which has been argued as a important parameter for the success of any reward program, or it can be attributed to price insensitivity of customers; whenever a customer is price insensitive, (s)he strictly prefers to purchase from the reward program merchant as (s)he gains points redeemable for rewards in the future.
There are many possible reasons for customers' price insensitivity: the reward program merchant could be offering some other monopoly products, or the customer might be getting reimbursed for some purchases as part of corporate perks (eg: corporate travel).
As an effect, this visit probability bias controls how frequently the customers' points increase even when (s)he does not actively choose to make purchases from the reward program merchant.
Both these parameters can be attributed to bounded rationality of customers and have been argued to be important factors toward customer choice dynamics, as discussed in the related work.


\subsubsection{Results Overview}
We formulate the customer choice dynamics as a dynamic program with the state being the number of points collected from the reward program merchant.
When the customer does not make biased visits to the reward program merchant, (s)he compares the immediate utility of purchasing the good at a cheaper price with the long term utility of waiting and receiving the time discounted reward, to make a choice. 
The solution to the customer's dynamic program gives conditions for the existence and achievability of a phase transition: a points threshold before which the customer visits the merchant offering rewards only due to the visit probability bias, and after which (s)he always visits the merchant offering rewards till receiving the reward.
We show that this phase transition occurs sooner for strategic customers, and with increase in the reward value offered by the reward program merchant.
It occurs later with increase in the points threshold required to redeem the reward, and with the increase in the price discount offered by the traditional pricing merchant.
With respect to the discount factor there is a unique minimizer for the phase transition point. That is, if customers are highly patient, then maximizing their long term utility leads them to redeem rewards only due to exogenous visit bias.
And if they are very less patient, they strictly prefer immediate discount over future rewards.
In short, these results verify that our model is in coherence with the different psychological constructs as discussed in the related work section: purchase acceleration closer to reward redemption and a tipping point before which purchases are only due to exogenous visit bias.

After characterizing the customer behavior dynamics in our model, we optimize over the long run revenues that the reward program merchant achieves.
We provide a framework for optimizing the parameters for the reward program merchant to maximize its revenue objective.
We then look at a specific simplified case of proportional promotion budgeting: the reward offered by the reward program merchant is proportional to the product of the distance to the reward and the discount provided by the traditional pricing merchant.
We show that under proportional promotion budgeting, the optimal distance to reward for maximizing the revenue objective of the reward program merchant is independent of the customer parameters and is close to the real world obtained cashback percentage amounts.
{\arpit In addition, optimizing the revenue objective gives the same optimal distance to reward as optimizing the influence zone as defined above.}
Moreover, we characterize the conditions in terms of the customer parameters for when the revenue objective of the reward program merchant is better than the traditional pricing merchant and when it is better for the reward program merchant to offer a reward versus not offering any reward.
We show that for the reward program to be effective under both the above conditions, a minimum fraction of customer population must be strategic.
And there is a range of values of the exogenous visit probability bias between $0$ and $1$ corresponding to the fraction of strategic customers for the reward program to be strictly better for the merchant. 
