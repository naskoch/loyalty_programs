We optimize the design of a frequency reward program against traditional pricing in a competitive duopoly, where customers measure their utility in rational economic terms.
We assume two kinds of customers: myopic and strategic (\cite{yilmaz2016upgrade}). 
Every customer has a prior loyalty bias (\cite{fader1993excess}) toward the reward program merchant, a parameter drawn from a known distribution indicating an additional probability of choosing the reward program merchant over the traditional pricing merchant.
This bias increases the switching costs (\cite{klemperer1995competition}) of strategic customers until a tipping point, after which they strictly prefer the reward program merchant. 
After characterizing this customer behavior, we optimize the reward parameters to maximize the revenue objective of the reward program merchant.
We show that the parameters for the reward program design, while optimizing the revenue objective, correspond exactly to minimizing the tipping point of customers, and are independent of the loyalty bias distribution parameter.
Moreover, it is essential for some fraction of the population to be strategic for the success of the reward program, with an optimal range for the loyalty bias parameter. 
If the bias is high, the reward program creates loss in revenues, as customers effectively gain rewards for ``free'', whereas a low value of bias leads to loss in market share to the competing merchant.
In short, if a merchant can estimate the customer population parameters, our framework and results provide theoretical guarantees on the pros and cons of running a reward program against traditional pricing.
