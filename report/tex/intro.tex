Loyalty programs constitute a huge market in consumer retail and are a major source of revenue for many low margin businesses.
Over 48 billion dollars in perceived value of rewards is issued in the United States alone every year, with every household having over 19 loyalty memberships on an average (\cite{berry2013loyalty}).
This market constitutes credit cards, hotel and airline reward programs, and more recently even restaurants, grocery and retail stores.
Though forming a big component of the market, there is little scientific understanding about the design of loyalty reward programs. We aim to address this gap with our research.

One popular form of loyalty reward programs is \emph{frequency reward programs}, where customers earn \emph{points} as currency over spendings with merchants and are able to redeem these points into dollar valued rewards after achieving certain threshold point collections.
There is extant literature on characterizing customer behavior toward frequency reward programs.
Most of the literature is empirical in nature, and relies on psychological behavioral patterns among customers, as opposed to rational economic decision making (\cite{kivetz2006goal,dreze2004using,gao2014influence}) .
On the merchant end, many benefits of running reward programs have been argued in literature in addition to possibly increasing the market share - for instance, user attribution for firms having multiple touchpoints; increase in sales due to referrals; personalized price discrimination and product recommendations, to name a few (\cite{ryu2007penny}).
Among a few examples of popular frequency reward programs, Starbucks allows members to earn ``stars'' on purchases which can be redeemed for free coffee drinks, Bloomingdale's offers \$25 reward for every 5000 points, and Target offers a 5\% cashback on all purchases (\cite{cvs2015target}).

In this paper, we consider a competitive duopoly of two merchants where one merchant offers a frequency reward program and the other offers traditional pricing with discounts.
Though revenue management literature often deals with dynamic pricing of products, many retail merchants offering rewards often pre-commit to their prices. 
We assume that both merchants commit to their product pricing apriori and characterize a novel model of customer choice where customers measure their utilities in rational economic terms.
In addition, we investigate the direct effects on the revenue objective of the merchant and characterize the optimal reward design choice for the merchant offering the frequency reward program, based on different customer populations.
Specifically, we answer the following question: how should the merchant decide the optimal thresholds and dollar value of rewards to optimize for its revenue share from the participating customer population.
One important constraint we impose is that the merchant has to choose a \emph{one design fits all} reward program for the entire participating customer population and is not allowed to personalize the program for different customer segments.

This is how the remaining of the paper is structured. First, we will describe some related work. Then we will give an overview of our model and results and justify some of our assumptions. 
In Section~\ref{sec:model} we will describe our model in technical detail followed by the main results in Section~\ref{sec:results}. 
We will conclude with a short discussion on future work in Section~\ref{sec:conc}.

\subsection{Related Work}
Three popular psychological constructs have been used to explain customer choice dynamics toward reward programs -- Goal Gradient Hypothesis, Medium Maximization, and Tipping Point Dynamics.
\cite{kivetz2006goal} conducted an empirical study observing an acceleration in the number of purchases by customers as they approached the reward, \ie, as customers accumulated reward points to reach closer to achieving the reward, their effort invested toward redemption increased. 
The authors attributed this behavior to Goal Gradient Hypothesis (\cite{hull1932goal}).
\cite{stourm2015stockpiling, dreze2004using} observed that customers often stockpiled reward points even when there were economic incentives against collection of points. They attributed this behavior to Medium Maximization -- customers often treated collecting reward points as a goal itself just like collecting stamps as opposed to connecting reward points with economic incentives.
Finally \cite{gao2014influence} observed via experimentation that customers often collect reward points for exogenous reasons until they accumulated a threshold amount, after which they started investing effort toward the collection process itself.
That is, customers build up switching costs before fully adopting the reward program, and sometimes this switching cost is built due to reasons exogenous to rational economic incentives.
They referred to this behavior as the Tipping Point Effect.

A large body of literature talks about the switching costs customers face within a competitive duopoly framework.
\cite{villas2015short} provides a short survey.
\cite{hartmann2008frequency} show using a similar model that only less frequent customers to a merchant face substantial switching costs.
\cite{kopalle2001economic} discuss dynamic competition between two merchants deciding on offering a reward program or traditional pricing and model this decision problem as a two stage game: first merchants decide whether to offer a reward program or traditional pricing and then they decide their prices. 
Depending on customer parameters in the model, they characterize the conditions for when it is better to offer a reward program versus traditional pricing.
They use customer's uncertainity about future preferences as a key component in their model attributing it to bounded rationality.
We on the other hand model a multi-period problem where the customer behavior is characterized using a full dynamic program.
We make two modeling assumptions into account: first is an exogenous visit probability bias toward the reward program merchant which can be attributed to \emph{excess loyalty} -- customers often build up higher brand preference toward the merchant offering a reward program (\cite{fader1993excess, sharp1997loyalty}); 
and second, customer myopicity, which indicates how far into the future customers can perceive the rewards (\cite{liu2007long,lewis2004influence}).
Our results on customer choice dynamics intuitively look similar to some of those obtained in this body of literature.
But more importantly, we model and optimize the revenue objective of the merchant, characterizing an optimal reward program setting for maximizing expected revenue.


\subsection{Our Contributions}
\subsubsection{Model Overview}
We model a competitive duopoly of two merchants, one of them offering a frequency reward program and the other offering traditional pricing.
Both merchants sell an identical good at fixed precommitted prices.
The reward program merchant sells the good at a higher price.
With each purchase from the reward program merchant, a customer gains some fixed number of points, and on achieving the reward redemption threshold, (s)he immediately gains the reward value as a dollar cashback.

Customers measure their utilities in rational economic terms, \ie, they make their purchase decisions to maximize long term discounted rewards.
The discount factor is the time value of money, and we assume it to be constant for all customers.
We also assume that every customer makes a purchase everyday from either of the two merchants.
We relax these two assumptions by introducing a look-ahead factor that controls how far into the future a customer can perceive the rewards. 
This affects the customer behavior dynamics as follows: if the reward is farther than the customer's look-ahead parameter, (s)he is unable to perceive the future value of that reward and take it into consideration while maximizing long term utility.
This parameter, being customer specific, adds heterogeneity to both the future discounting and purchase frequency.
We only model myopic and strategic customers, \ie, the look-ahead parameter being $0$ or a large value, and leave further parametrization for future work.
{\nolan I'm looking for source about different levels of myopicity and then I will make this a little stronger.}
In addition, we assume each customer has a visit probability bias with which (s)he purchases the good from the reward program merchant for reasons exogenous to utility maximization.
This can be attributed to \emph{excess loyalty} (\cite{fader1993excess, sharp1997loyalty}) which has been argued as a important parameter for the success of any reward program, or it can be attributed to price insensitivity of customers; whenever a customer is price insensitive, (s)he strictly prefers to purchase from the reward program merchant as (s)he gains points redeemable for rewards in the future.
There are many possible reasons for customers' price insensitivity: the reward program merchant could be offering some other monopoly products, or the customer might be getting reimbursed for some purchases as part of corporate perks (eg: corporate travel).
As an effect, this visit probability bias controls how frequently the customers' points increase even when (s)he does not actively choose to make purchases from the reward program merchant.
Both these parameters can be attributed to bounded rationality of customers and have been argued to be important factors toward customer choice dynamics, as discussed in the related work.


\subsubsection{Results Overview}
We formulate the customer choice dynamics as a dynamic program with the state being the number of points collected from the reward program merchant.
When the customer does not make biased visits to the reward program merchant, (s)he compares the immediate utility of purchasing the good at a cheaper price with the long term utility of waiting and receiving the time discounted reward, to make a choice. 
The solution to the customer's dynamic program gives conditions for the existence and achievability of a phase transition: a points threshold before which the customer visits the merchant offering rewards only due to the visit probability bias, and after which (s)he always visits the merchant offering rewards till receiving the reward.
We show that this phase transition occurs sooner for strategic customers, and with increase in the reward value offered by the reward program merchant.
It occurs later with increase in the points threshold required to redeem the reward, and with the increase in the price discount offered by the traditional pricing merchant.
With respect to the discount factor there is a unique minimizer for the phase transition point. That is, if customers are highly patient, then maximizing their long term utility leads them to redeem rewards only due to the loyalty bias.
And if they are very less patient, they strictly prefer immediate discount over future rewards.
In short, these results verify that our model is in coherence with the different psychological constructs as discussed in the related work section: purchase acceleration closer to reward redemption and a tipping point before which purchases are only due to the loyalty bias.

After characterizing the customer behavior dynamics in our model, we optimize over the long run revenues that the reward program merchant achieves.
We model a specific case of proportional promotion budgeting: the reward offered by the reward program merchant is proportional to the product of the distance to the reward and the discount provided by the traditional pricing merchant, the proportionality constant being another parameter in the design of the reward program.
We show that under proportional promotion budgeting, the optimal distance to reward and the proportionality budgeting constant follow an intuitive product relationship which is independent of the customer population parameters.
And they correspond closely to real world observed cashback percentage amounts.
In addition, optimizing the revenue objective gives the same optimal distance to reward as minimizing the phase transition point as defined above.
Moreover, we characterize the conditions in terms of the customer parameters for when the revenue objective of the reward program merchant is better than the traditional pricing merchant and when it is better for the reward program merchant to offer a reward versus not offering any reward, for a specific choice of loyalty bias distribution.
We show that for the reward program to be effective under both the above conditions, a minimum fraction of customer population must be strategic.
And there is a range of values of the loyalty bias between $0$ and $1$ corresponding to the fraction of strategic customers for the reward program to be strictly better for the merchant. 

