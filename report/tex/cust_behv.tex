\subsection{Customer Choice Dynamics}
We first show that every customer exhibits the following behavior: until (s)he reaches the phase transition point $i_0(t)$, she purchases from $A$ only due to the exogeneity paramater, and after that (s)he always purchases from $A$ till she receives the reward.
This behavior is cyclic, and repeats after every reward redemption.

\begin{lemma} $V(i)$ is an increasing function in $i$ if the following condition holds:
\begin{equation}
R > \frac{(1-\lambda)v}{1-\beta}
\end{equation}
And further, $V(i)$ can be evaluated as:
\begin{equation}
V(i) = \max\left\{ \frac{\lambda \beta V(i+1)+(1-\lambda)v}{1-(1-\lambda)\beta}, \beta V(i+1) \right\}
\end{equation}
\end{lemma}

\proof
First we show that $V(i)$ is an increasing function in $i$ by induction. We first show that if the condition above is satisfied, $V(k-1) < V(k) = R$. Suppose not, so $V(i) \geq R$. Then we have:
\begin{align*}
V(k-1) &= \lambda \beta V(k) + (1-\lambda)(v+\beta V(k-1)) \\
&= \frac{\lambda \beta R + (1-\lambda)v}{1-(1-\lambda)\beta} \\
&< \frac{\lambda \beta R + (1-\beta)R}{1-(1-\lambda)\beta} \\
&= \frac{R(1-(1-\lambda)\beta)}{1-(1-\lambda)\beta} = R
\end{align*}
But this is a contradiction, so $V(k-1) < V(k)$. Now assume $V(i+1) < V(i+2)$ for some $i < k-2$, we will show that this implies $V(i) < V(i+1)$. Suppose not, so $V(i) \geq V(i+1)$. As we did before we may upper bound $V(i)$.
\begin{align*}
V(i) &= \lambda \beta V(i+1) + (1-\lambda)(v+\beta V(i)) \\
&\leq (1-\lambda)v + \beta V(i) \\
\iff V(i) &\leq \frac{(1-\lambda)v}{1-\beta}
\end{align*}
But because $V(i+1) < V(i+2)$, we may lower bound $V(i+1)$.
\begin{align*}
V(i+1) &\geq \lambda \beta V(i+2) + (1-\lambda)(v+\beta V(i+1)) \\
&= (1-\lambda)v + (1-\lambda)\beta V(i+1) + \lambda \beta V(i+2) \\
&> (1-\lambda)+\beta V(i+1) \\
\iff V(i+1) &> \frac{(1-\lambda)v}{1-\beta}
\end{align*}
Again, we have a contradiction, so $V(i) < V(i+1)$, and $V(i)$ is an increasing function in $i$. Now we prove the second claim. We have the following:
\begin{align*}
V(i) &= \lambda \beta V(i+1) + (1-\lambda)\max\{v +\beta V(i), \beta V(i+1) \} \\
&= \max\{\lambda \beta V(i+1) + (1-\lambda)(v+\beta V(i)), \beta V(i+1) \}
\end{align*}

Assuming $V(i)$ is the left term in the above maximum, we may solve the equation for that term.
\begin{gather*}
V(i) = \lambda \beta V(i+1) + (1-\lambda)(v+\beta V(i)) \\
(1-(1-\lambda)\beta) V(i) = \lambda \beta V(i+1) + (1-\lambda)v \\
V(i) = \frac{\lambda \beta V(i+1) + (1-\lambda)v}{1-(1-\lambda)\beta}
\end{gather*}
And we get our claim.
\endproof

Now if the expected reward of the customer increases with the number of purchases made from $A$, we expect that at some number of purchases it becomes profitable for the customer to choose to purchase from $A$ as opposed to $B$.
We characterize this phase transition point in the following theorem.

\begin{theorem} Suppose $V(i)$ is an increasing function in $i$ and consider a customer with look-ahead parameter $t$. A phase transition occurs after (s)he makes $i_0(t)$ visits to firm $A$, where $i_0(t)$ is given by:
\begin{equation}
  i_0(t)=\begin{cases}
    k-\Delta \equiv i_0, & \text{if $t \geq \Delta$}.\\
    k-t, & \text{otherwise}.
  \end{cases}
\end{equation}
with 
\begin{align}
\Delta &= \left\lfloor \log_{\beta}\left(\frac{v}{R(1-\beta)}\right)\right\rfloor
\end{align}
\end{theorem}

\proof
First we solve for the condition on $V(i+1)$ for us to choose firm $A$ over $B$ willingly.
\begin{gather*}
\beta V(i+1) > \frac{\lambda \beta V(i+1) + (1-\lambda)v}{1-(1-\lambda)\beta} \\
\iff \beta V(i+1) \left(1-\frac{\lambda}{1-(1-\lambda)\beta} \right) > \left(\frac{1-\lambda}{1-(1-\lambda)\beta} \right) v \\
\iff \beta V(i+1) \left(\frac{1-(1-\lambda)\beta -\lambda}{1-(1-\lambda)\beta} \right) > \left(\frac{1-\lambda}{1-(1-\lambda)\beta} \right) v \\
\iff \beta V(i+1) \left(\frac{(1-\lambda)(1-\beta)}{1-(1-\lambda)\beta} \right) > \left(\frac{1-\lambda}{1-(1-\lambda)\beta} \right) v \\
\iff \beta V(i+1) > \frac{v}{1-\beta} \\
\iff V(i+1) > \frac{v}{\beta(1-\beta)}
\end{gather*}
Let $i_0$ be the minimum state $i$ such that the above holds, so in particular $V(i_0) \le \frac{v}{\beta(1-\beta)}$ but $V(i_0+1) > \frac{v}{\beta(1-\beta)}$. We know because $V$ is increasing in $i$, this point is indeed a phase transition: $V(i) > \frac{v}{\beta(1-\beta)}$ for all $i > i_0$, so after this point, the customer always chooses firm $A$. We may compute $V(i_0)$ easily using this fact.
\begin{equation*}
V(i_0) = \beta V(i_0+1) = \cdots = \beta^{k-i_0}V(k) = \beta^{k-i_0}R
\end{equation*}
Thus, we have the following:
\begin{gather*}
\beta^{k-i_0} \le \frac{v}{R\beta(1-\beta)} < \beta^{k-(i_0+1)} \\ 
\iff k-i_0 \ge \log_{\beta}\left(\frac{v}{R\beta(1-\beta)} \right) > k-(i_0+1) \\
\iff i_0 \le k - \log_{\beta}\left(\frac{v}{R(1-\beta)} \right) + 1 < i_0 + 1\\
\iff i_0 = k - \left\lfloor \log_{\beta}\left(\frac{v}{R(1-\beta)}\right) \right\rfloor \equiv k-\Delta
\end{gather*}

The above dependence reduces to the following after incorporating the look-ahead distribution:

\begin{equation*}
  i_0(t)=\begin{cases}
    i_0, & \text{wp } p,\\
    k, & \text{wp } 1-p.
  \end{cases}
\end{equation*}
\endproof

Note that the phase transition point is independent of $\lambda$, the customer's visit probability bias toward the merchant.
As we would expect, it increases with the look-ahead parameter, and with the price discount offered by merchant $B$.
Additionally, it decreases with an increase in the reward value $R$ and a decrease in the distance to reward $k$.
The variation with the discount factor $\beta$ is interesting: we can show that for any $\frac{R}{v} \ge 1$ there exists a $\beta \in [0,1]$ that minimizes the phase transition point $i_0$ for ``forward-looking'' customers.
This means that customers who are more patient have a longer time frame to transition and so do customers who are less patient than the optimal value.
We refer to the ratio of number of visits required for a forward-looking customer to adopt a reward program and the total distance to the reward as the ``influence zone''.
Intuitively this is the fraction of visits that the merchant wants to influence the customer by offering exogenous means of earning additional points like bonus miles in airlines, or accelerated earnings, as discussed in the introduction.
Next we find the optimal $k$ for minimizing this influence zone when the proportional promotion budgeting parameter $\alpha$ is constant.

\begin{remark}\label{rem:inf_zone}
Influence zone is minimized at $k = \frac{e}{\alpha(1-\beta)}$ under constant proportional promotion budgeting. 
\end{remark}
\proof
As defined the influence zone is $\frac{i_0}{k} = \frac{k-\Delta}{k} = 1 -\frac{\Delta}{k}$.
Thus minimizing the influence zone is equivalent to maximizing $\frac{\Delta}{k}$.
\begin{align*}
\frac{\Delta}{k} = \frac{\log_\beta\left(\frac{1}{\alpha k(1-\beta)}\right)}{k}\sim \frac{\log (\alpha k(1-\beta))} {k(1-\beta)}
\end{align*}
The above approximation relies on $\beta$ close to 1. Now this value is maximized at $k = \frac{e}{\alpha(1-\beta)}$. Therefore, for all $b$, the optimal value for $k$ is given by $\frac{e}{\alpha(1-\beta)}$, the value for which $\frac{\Delta}{k}$ is maximized and takes the value $\frac{\alpha}{e}$. 
At this value the influence zone takes the value $1-\frac{\alpha}{e}$.
\endproof
Note that if $\alpha$ is $1$, then the value of $k$ corresponds to a cashback between $2$\% and $4$\% as $\beta$ ranges between $0.95$ and $0.9$.
This value is realistic to what is observed in practice.
