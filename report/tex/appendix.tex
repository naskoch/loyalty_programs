\newpage
\section{Appendix}

\subsection{Proof of Lemma~\ref{lem:lower_b}}
\proof{Proof.}
We delay the proof of this lemma to first prove a helpful proposition. It is a straightforward computation to see that the condition of $RoR_A \geq \frac{b}{2}$ is equivalent to:
\begin{gather*}
\frac{1}{b}\left(1-\frac{e-\alpha}{b\alpha}\log \left(1+\frac{b\alpha}{e-\alpha} \right) \right) \geq \frac{\alpha(1-(1-p)(1-\alpha v))}{2pe(1-\alpha v)} \\
\iff
g(b; \alpha) \geq h(p, v; \alpha)
\end{gather*}
where we have defined functions $g(b)$ and $h(p,v)$ for fixed $\alpha$ for the above inequalities. 

\begin{proposition}
For a fixed $\alpha$, $g(b)$ is non-increasing for all $b \in (0,1)$. 
\end{proposition}

\proof{Proof.}
We take the derivative of $g$:
\begin{align*}
g'(b) &= \frac{2(e-\alpha)}{b^3 \alpha} \log\left(1+\frac{b\alpha}{e-\alpha} \right) - \frac{1}{b^2}-\frac{1}{b^2\left(1+\frac{b\alpha}{e-\alpha}\right)} \leq 0 \\
&\iff \frac{2(e-\alpha)}{b \alpha} \log\left(1+\frac{b\alpha}{e-\alpha} \right) \leq 1+\frac{1}{1+\frac{b\alpha}{e-\alpha}} \\
&\iff \frac{2\log(1+x)}{x} \leq 1+\frac{1}{1+x}
\end{align*}
where $x = \frac{b\alpha}{e-\alpha}$, and as $b \in (0,1)$, $x \in (0, \frac{\alpha}{e-\alpha})$. We can see that as $x \rightarrow 0$, the above inequality is an equality. 
We represent the LHS of the above equation as $L(x)$ and RHS as $R(x)$. Next we show that $L(x)$ decreases more quickly than $R(x)$ does for positive $x$, thereby proving the proposition.
First show that in the range of $x$ the following holds true:

\beq
\label{eq:eq00}
\left(2-\frac{1}{1+x}\right)^2 \le 2\log(1+x) + 1
\eeq
To show the above observe that at $x\rightarrow 0$ both the LHS and RHS are equal. And it is easy to show that the derivative of LHS is lower than the derivative of RHS for all $x\ge 0$ as shown.
\begin{align*}
& (1+x) + \frac{1}{1+x} \ge 2\\
\implies & 2 - \frac{1}{1+x} \le 1 + x\\
\implies & \left(2-\frac{1}{1+x}\right)\cdot \frac{1}{1+x} \le 1\\
\implies & 2\cdot \left(2-\frac{1}{1+x}\right)\cdot \left(\frac{1}{1+x}\right)^2 \le \frac{2}{1+x}
\end{align*}
The left hand side is the derivative of the above LHS and right hand side is the derivative of the above RHS.

Now we can rearrange Eq.~\ref{eq:eq00} as follows:
\begin{align*}
& \left(2-\frac{1}{1+x}\right)^2 \le 2\log(1+x) + 1\\
\implies & \left(1 + \frac{x}{1+x}\right)^2 \le 2\log(1+x) + 1\\
\implies & \left(\frac{x}{1+x}\right)^2 + \frac{2x}{1+x} \le 2\log(1+x)\\
\implies & 2\left(\frac{x}{1+x} - \log(1+x)\right) \le - \left(\frac{x}{1+x}\right)^2\\
\implies & \frac{2\left(\frac{x}{1+x} - \log(1+x)\right)}{x^2} \le - \left(\frac{1}{1+x}\right)^2
\end{align*}
The left hand side of above is $L'(x)$ and right hand side is $R'(x)$.

\endproof

\subsection{Proof of Lemma~\ref{lem:upper_b}}
\proof{Proof.}
Let $\frac{b\alpha}{e-\alpha} = x$. Then $RoR_A > RoR_B$ can be evaluated as follows:

\begin{eqnarray}
& p\frac{e}{\alpha}\left(1-\frac{\log(1+x)}{x}\right)(1-\alpha v + 1 - v) - p(1-v) + (1-p)\frac{b}{2}\left(1-\alpha v + 1-v\right) + p(1-v) > 1-v\notag\\
& \implies p\left(1 - \frac{\log(1+x)}{x}\right) + (1-p)\frac{b\alpha}{2e} > \frac{\alpha}{e} \cdot \frac{1-v}{1-\alpha v + 1 - v}\label{eq:ra>rb}
\end{eqnarray}

Since $\alpha$ is a constant, the LHS above is a function of $b$ and $p$. 
Let the LHS above be $L(b,p)$.
We first show that in the range of $b\in [0,1]$, $1 - \frac{\log(1+x)}{x} > \frac{b\alpha}{2e}$ which shows that $L(b,p)$ is increasing in $p$.

\begin{align*}
& 1-\frac{\log(1+x)}{x} > \frac{b\alpha}{2e}\\
\Leftrightarrow & x - \log(1+x) > \frac{b^2\alpha^2}{2e(e-\alpha)}
\end{align*}
Observe that LHS is equal to RHS when $b\rightarrow 0$. 
All we show is that LHS increases faster than RHS in the range of $b\in [0,1]$. 

\begin{align*}
\Leftrightarrow & \left(1 - \frac{1}{1+x}\right) \frac{\alpha}{e-\alpha} > \frac{b\alpha^2}{e(e-\alpha)}\\
\Leftrightarrow & \frac{1}{1+x} > \frac{e-\alpha}{e}\\
\Leftrightarrow & \frac{e}{e-\alpha} > 1 + \frac{b\alpha}{e-\alpha}
\end{align*}
And the last equation is true in the range of $b\in [0,1]$. Hence $L(b,p)$ increases with $p$.

Now we show that $L(b,p)$ increases with $b$ as well. First observe:

\beq
\frac{\partial L(b,p)}{\partial b} = p\left(\frac{\log(1+x) - \frac{x}{1+x}}{x^2}\right)\frac{\alpha}{e-\alpha} + (1-p)\frac{\alpha}{2e}
\notag
\eeq

Thus $\frac{\partial L(b,p)}{\partial b} > 0$ implies:

\begin{align*}
& (1-p)\frac{\alpha}{2e} > p\left(\frac{\frac{x}{1+x} - \log(1+x)}{x^2}\right)\frac{\alpha}{e-\alpha}\\
\Leftrightarrow & (1-p)\frac{b^2\alpha^2}{2e(e-\alpha)} > p \left(1 - \frac{1}{1+x} - \log(1+x)\right)
\end{align*}

Again the LHS and RHS are equal as $b\rightarrow 0$. All we show again is that LHS increases faster as compared to RHS.

\begin{align*}
\Leftrightarrow (1-p)\frac{b\alpha^2}{e(e-\alpha)} > p\left(\frac{1}{(1+x)^2} - \frac{1}{1+x} \right)\frac{\alpha}{e-\alpha}\\
\end{align*}
Clearly RHS is negative when $b\in (0,1]$ and LHS is positive. Hence proved.

Thus $L(b,p)$ is increasing in both $b$ and $p$. And the condition required is $L(b,p)$ is greater than some constant value which depends on $v$.
Hence for any $v$ there exists a smooth $(b_0,p_0)$ curve such that for all $b\ge b_0$ and $p\ge p_0$ revenue rate of reward program merchant is larger.

\endproof

