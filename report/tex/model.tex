Consider a competitive duopoly of two stores, $A$ and $B$, selling the same item. 
Without loss of generality we assume that store $A$ sells the item for a price of $1$ dollars while store $B$ sells it for $1-v$ dollars, \ie, $B$ offers a discount of $v$ dollars. 
Store $A$ on the other hand offers a reward of value $R$ dollars to a customer after (s)he makes $k$ purchases at $A$. 
Our goal is to understand the dynamics of competition between $A$ and $B$ with respect to setting of the reward and discount parameters to maximize their respective revenues over a distribution of customers under certain assumptions on their behavior.

\subsection{Customer Choice Model}
First we assume that every customer purchases an item from either $A$ or $B$ everyday.
We assume that there is some exogenous probability, $\lambda$, during each purchase that forces the customer to go to store $A$.
This $\lambda$ is a customer specific parameter and is drawn from a uniform distribution between $[0,b]$, where $b$ is between $0$ and $1$.
Let $0 < \beta \leq 1$ denote the discounting factor of future money. 
We assume customers to have a linear homogenous utility in price: at price $p$ the utility is $\nu(p) = 1-p$. 
This reduces to customers getting an immediate utility of $0$ from $A$ and $v$ from $B$.

We model the customer's decision problem as a dynamic problem. We index the number of visits the customer makes at store $A$ by $i$, for $0 \leq i \leq k-1$, and we refer a customer to be in state $i$ after having made $i$ visits to $A$. At state $i$, the customer has two possibilities:
\begin{enumerate}
\item
With probability $\lambda$, the customer must visit $A$, and she is now in state $i+1$.
\item
With probability $1-\lambda$, the customer may purchase from $B$ for an immediate utility $v$ and remain in state $i$ or purchase from $A$ for no utility but move to state $i+1$.
\end{enumerate}

Let $V(i)$ denote the long term expected reward at state $i$. Then we may model the decision problem as the following dynamic program.
\begin{align*}
& V(i) = \lambda \beta V(i+1) + (1-\lambda)\max\{v+\beta V(i),\beta V(i+1) \} \mbox{ for } 0\leq i \leq k-1 \\
& V(k) = R
\end{align*}

We will show that the decision process exhibits a phase transition; that is prior to some state $i_0$, the customer will only visit $A$ if she must do so exogenously but after $i_0$, she always decides to go to $A$. 

Finally, we assume the customer has a look-ahead factor $t$, which models how many purchases ahead the customer looks ahead when making her current decision. 
This value will affect the phase transition of the decision process. 
Consider a distribution $T$ describing the look-ahead factor for consumers. 
We will focus on threshold distributions; for example, with probability $p$ the look-ahead is $t_1$ and with probability $1-p$, the look-ahead is $t_0$.
With the addition of this look-ahead factor, the phase transition point will now depend on it, and we will refer to it as $i_0(t)$.

\subsection{Merchant Objective and Competition}
Given the above model of customer dynamics, and that customer exogeneity and look ahead parameters are drawn from a known distribution, the two merchants try competing over the customer base to maximize their long run revenues.
We define the rate of revenue for a merchant from a customer as the expected time averaged revenue that the merchant receives within the customer's lifetime.
For simplification we assume merchants do not discount future revenues.
As described above, a customer's dynamics are cyclic after each reward cycle.
Thus the lifetime dynamics of customer behvior is a regenerative process with independent and identically distributed reward cycle lengths.
Let $RoR_A(c)$ and $RoR_B(c)$ denote the expected rate of revenues for merchants $A$ and $B$ respectively from a customer $c$'s lifetime.
Let $\tau(i_0(t))$ denote the total number of visits the customer makes before reaching the phase transition point $i_0(t)$.
Then the length of the reward cycle (or total number of visits the customer makes before receiving the reward) is $\tau(i_0(t)) + k - i_0(t)$, as after the phase transition (s)he makes all visits to merchant $A$ only till hitting the reward.
In this cycle the number of visits that the customer makes to $A$ are $k$, and to $B$ are $\tau(i_0(t)) - i_0(t)$.
Thus the rate of revenues are as follows:
\beq
RoR_A(c) = \underset{\tau}E\left[\frac{k-R}{\tau(i_0(t)) + k - i_0(t)}\right]
\eeq
\beq
RoR_B(c) = \underset{\tau}E\left[\frac{(\tau(i_0(t)) - i_0(t))(1-v)}{\tau(i_0(t)) + k - i_0(t)}\right]
\eeq

The goal of each merchant is to set their reward or pricing parameters so as to maximize the expected value of rate of revenue over the entire customer population.
Before that, since the process for a single customer is regenerative, using the reward renewal theorem (CITE), we can take the expectation over the cycle length inside the numerator and denominator respectively.
Note that $\underset{\tau}E[\tau(i_0(t))] = \frac{i_0(t)}{\lambda}$ as before reaching the phase transition point, with probability $\lambda$, the customer's visits to $A$ increases by $1$ and with probability $1-\lambda$ it stays constant.
Then the objectives of the two merchants are:

\beq
\underset{R,k}\max\left\{RoR_A\right\} = \underset{R,k}\max\left\{\underset{\lambda, t}E\left[\frac{k-R}{i_0(t)/\lambda + k - i_0(t)}\right]\right\}
\eeq

\beq
\underset{v}\max\left\{RoR_B\right\} = \underset{v}\max\left\{\underset{\lambda, t}E\left[\frac{(\tau(i_0(t)) - i_0(t))(1-v)}{i_0(t)/\lambda + k - i_0(t)}\right]\right\}
\eeq
